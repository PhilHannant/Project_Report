\documentclass[a4paper, 11pt]{article}
\setlength{\topmargin}{0.0in}
\setlength{\textheight}{9.5in}
\setlength{\oddsidemargin}{-.1in}
\setlength{\textwidth}{6.5in}
\usepackage{parskip}
\usepackage{graphicx}
\graphicspath{ {images/} }

\usepackage{tocbibind}
\usepackage[toc,page]{appendix}
\usepackage{datetime}
\usepackage[font={small,it}]{caption}
\usepackage{listings}
\usepackage[font={small,it}]{subcaption}
\usepackage{amssymb,amsmath}

\newdateformat{monthyeardate}{%
  \monthname[\THEMONTH], \THEYEAR}

\date{}
\begin{document} 

\LARGE\title{Real Time Tempo Analysis of Drum Beats}

\LARGE\author{Author: \textbf{Philip Hannant}, Supervisor: \textbf{Professor Steve Maybank}\\
\\Birkbeck, University of London\\
Department of Computer Science and Information Systems\\
\\Project Report\\
MSc Computer Science\\
\\\monthyeardate\today
}





\normalsize


\maketitle
\newpage
\tableofcontents
\clearpage

\section*{Abbreviations}
\begin{tabular}{l p{4.5in}  }\\
\textbf{BPM} & Beats Per Minute\\
\textbf{DFT} & Discrete Fourier Transform\\
\textbf{DWT} & Discrete Wavelet Transform\\
\textbf{FFT} & Fast Fourier Transform\\
\textbf{JSON} & JavaScript Object Notification\\
\textbf{SBT} & Simple Build Tool\\
\textbf{TDD} & Test Driven Development\\
\end{tabular}

\section*{Definitions}
\begin{tabular}{l p{4.5in}}\\
\textbf{Acoustic Drum Kit} & A collection of drums and cymbals which do not have electronic amplification. Typically made up of a bass drum, snare drum, tom-toms, hi-hat and 1 or more cymbals.\\\\
\textbf{Beat} & For the purpose of this project a beat will be defined as the sequence of equally spaced pulses used to calculate the tempo being played by the drummer.\\\\
\textbf{Drum Module} & The device which serves as a central processing unit for an electronic drum kit, responsible for producing the sounds of the drum kit.\\\\
\textbf{Electronic Drum Kit} & An electrical device which is played like an acoustic drum kit, producing sounds from a stored library of instruments and samples.\\\\
\textbf{MIDI} & Musical Instrument Digital Interface is a protocol developed in the 1980's to allow electronic instruments and other digital musical tools to communicate with each other \cite{midi}.\\\\
\textbf{Tempo} & The speed at which a piece of music is played \cite{oxford-comp} and counted in beats per minute (bpm).\\\\
\end{tabular}
\clearpage

\maketitle{} \section{Introduction and Background}

This project report presents my aim to develop a real-time drum beat tempo analysis system using different beat detection algorithms which is able to record the performance of each method when an extensive set of drum samples, representing a real drummer's performance, is processed through the system.

\subsection{Drumming Training Tools Background}
Timing is the fundamental skill any good drummer should possess and is the staple by which they will be judged. For many years the only training tool available to a drummer to improve their timing was the metronome. An instrument used to mark musical tempo, erroneously attributed to Johann Nepomuk Maelzel in 1815 but was actually invented by a Dutchman, Dietrich Nikolaus Winkel a year earlier. The traditional metronome, based on Winkel's original design is a hand-wound clockwork instrument that uses a pendulum swung on a pivot to generate the ticking which depicts the desired tempo \cite{brit-metro} is still used today by musicians, as seen in Figure~\ref{fig: TradMet}. \par

\begin{figure}[ht]
	\centering
	\includegraphics[scale=0.25]{TradMet}
	\caption{Traditional Metronome}% (Source: https://commons.wikimedia.org/w/index.php?curid=1058005)}
	\label{fig: TradMet}
\end{figure}

For drummers however the electronic versions of the metronome are much more widely used, to the point that metronomes are now developed with functionality specifically tailored to a drummers training requirements. The Tama Rhythm Watch (Figure~\ref{fig: RW200}) was the first metronome designed specifically for drummers, providing enough volume to be used with real drums as well as allowing for the use of different time signatures\footnote{A details explanation of time signatures can be found in section 1.2.2} and preset set rhythm patterns to help improve performance.\par

\begin{figure}[htbp]
	\centering
	\includegraphics[scale=0.2]{RW200}
	\caption{Tama Rhythm Watch}% (Source: http://www.tama.com/eu/products/accessories/RW200.html)}
	\label{fig: RW200}
\end{figure}

Following the development of the MIDI driven electronic drum kit came the development of more advanced training tools that were able to provide live feedback to drummer during any given performance. Today the leaders in this field are Roland, their V-Drums line provide a variety of tuition packages including the SCOPE and more recently the COACH system provided in the V-Drum modules. The SCOPE system provides live feedback to the drummer by plotting their performance on a grid representing the elements of the drum kit, Figure~\ref{fig: scope}. The COACH system offers an improved user experience by providing an accuracy figure relating to the number of hits which the drummer has played in time, Figure~\ref{fig: coach}.
\begin{figure}[ht]
\centering
\begin{subfigure}{.5\textwidth}
  \centering
  \includegraphics[width=0.5\linewidth]{images/Scope.jpg}
  \caption{Screen shot of the SCOPE system}
  \label{fig: scope}
\end{subfigure}%
\begin{subfigure}{.5\textwidth}
  \centering
  \includegraphics[width=0.75\linewidth]{images/coach.jpg}
  \caption{Screen shot of the COACH system}
  \label{fig: coach}
\end{subfigure}
\caption{SCOPE and COACH training tools}
\label{fig: systems}
\end{figure}

The V-Drum Rhythm Coach line is an advanced version of the traditional drummers practice pad (Figure~\ref{fig: rmp-5} and the extensive DT-1 V-Drums tutor software package (Figure~\ref{fig: dt-1}. 
\begin{figure}[ht]
\centering
\begin{subfigure}{.5\textwidth}
  \centering
  \includegraphics[width=0.5\linewidth]{images/rmp-5.jpg}
  \caption{The RMP-5 Rhythm coach practice pad}
  \label{fig: rmp-5}
\end{subfigure}%
\begin{subfigure}{.5\textwidth}
  \centering
  \includegraphics[width=0.75\linewidth]{images/dt-1_ss_main_notation_gal.jpg}
  \caption{Screen shot of the DT-1 V-Drums Tutor}
  \label{fig: dt-1}
\end{subfigure}
\caption{}
\label{fig: roland systems}
\end{figure}
Roland have even used gamification within this field with their latest release, the V-Drums Friend Jam app. The application itself provides the player with live feedback and evaluates each performance in order to provide the player with a score which they can share over social media, a screen shot can be seen in Figure~\ref{fig: friendjam}. \par

\begin{figure}[htbp]
\centering
\includegraphics[scale=0.175]{images/friendjam.png}
\caption{Example of the Roland Friend Jam software package}
\label{fig: friendjam}
\end{figure}

Despite these developments there have been a limited number of tools available to be used with acoustic drums. A look at the the App Store\footnote{Apple's application store - itunes.apple.com/uk/appstore‎} and Google Play\footnote{Google's application store for Android devices - play.google.com/store}, reveals a small number of general live bpm detectors an no applications which are focused towards a drummers training needs.\par

\subsection{Drum Musical Theory}
In order to understand the fundamentals of musical timing some theory needs to be examined but beforehand the concept of a drummer playing time must be considered. Time, in a drumming sense is an informal term used to describe the consistent rhythmic pattern that a drummer will play on the hi-hat or ride cymbal \cite{drum-bible} and it can be considered one of the most important components of any drum beat. 

\subsubsection{Notation}
Drum music notation is written on staff that is made up of five individual lines, the clef is found on the far left of the staff which indicates the pitch of the notes \cite{oxford-comp} and as percussion instruments are non-pitched they use the percussion-clef. On traditional musical notation the lines and spaces between represent a tonal where as for drum notation, notes written on lines or spaces indicate a certain drum or cymbal. The staff is separated into individual measures which are known as bars \cite{drum-note} and it is these bars that are the basis of musical time, an example of a staff can be seen in Figure~\ref{fig: staff}. For the purpose of this project it is the count of these beats that will be used to calculate the tempo of a certain drum beat.

\begin{figure}
	\centering
	\includegraphics[scale=0.3]{images/staff.jpg}
	\caption{Example of a staff used in musical notation}
	\label{fig: staff}
\end{figure}

\subsubsection{Time Signatures}
Time signatures appear on the staff just after the clef and are written as a fraction where the top number indicates the number of beats that there are in a bar, as seen in Figure~\ref{fig: staff}. With the bottom number representing the size of the note that makes up the duration of one beat. For example the straight time four four (4/4) or common time signature indicates four beats in each bar or measure where each beat is made up of one quarter note \cite{drum-note}. Within these bar lines beats can be further divided by using a technique known as subdivision, which is a method for reducing the pulse or rhythm pattern into smaller parts than those originally written, for example counting a four four (4/4) measure in eighth (1/8) or sixteenth (1/16) notes. 

\subsubsection{Notes}
The notes used to represent what percussive instrument is to be played also provide the duration it should be played for. Notes come in different lengths and the key values are the whole (1/1), half (1/2), quarter (1/4), eighth (1/8) and sixteenth (1/16). For example two eighth notes represent the same time value as a single quarter note. It is also possible to divide note values by three instead of two, these notes are known as triplets. An eighth note triplet is played fifty percent faster than a normal eighth note, therefore for every two eighth notes there will be three eighth note triplets \cite{drum-note}. An example of the eighth-note triplets being used is in a twelve eight (12/8) jazz shuffle, the time element played on the ride cymbal or hi-hat is charaterised by playing the first and third triplet of an eighth-note triplet grouping \cite{drum-bible}.

\subsubsection{Playing Basics}
With the basics of drum theory covered it is now possible to discuss the key elements of a drum beat, typically for a straight four four (4/4) beat, the bass drum will be played on the first and third beat and the snare drum will be played on the second and forth beat both as quarter notes. This is more commonly known as a back beat \cite{drum-bible}. This just leaves the time element which will usually be played on the ride cymbal or hi-hat, this too could be played using quarter notes on the first, second, third and forth beats. However, in order to make the drum pattern more dynamic the time element will usually be played using subdivisions, typically using eighth note subdivisions. The ride cymbal or hi-hat will therefore be played on the first, second, third and forth beats as well as the eighth notes in-between each quarter note. This can be demonstrated by counting the one-and-two-and-three-and-four-and, where the ``and'' represents the subdivided eighth note. Playing time on the ``and'' note is also known as playing on the off beat. Additionally to this technique a drummer will usually ensure that there is difference in the volume of the eighth notes being played on the quarter notes and those being played on the ``and''. This technique of emphasising certain beats is known as accenting. 


\subsection{Beat Detection Background}
Most of the early work on beat detection was a by-product of research directed at other areas of musical understanding. The earliest work in this field can be attributed to H. C. Longuet-Higgins, who in 1976 while researching the psychological theory of how Western musicians perceive rhythmic and tonal relationships between notes. Produced an algorithm that was able to follow the beat of a performance and adjust the perceived tempo accordingly based on whether a note started earlier or later than expected \cite{allen-danneburg}. Longuet-Higgins' work was built on the premise that in order to perceive the rhythmic structure of a melody it is first necessary to identify the time at which each beat occurs \cite{longeut1}, otherwise know as onset detection. The onset of a note is the instant which marks the start of the variation in the frequency of a signal, a visualisation of this can be seen in Figure~\ref{fig: Onset}. Once detected it can then be used to measure the onset times of sonic events\footnote{A sonic event is a singular feature of a piece music which can be made up of one source or many \cite{sonic}, e.g. the hitting of a drum} within a piece of music \cite{mirex-onset}. These onset times are then used within a beat detection algorithm in order to calculate a piece of music's tempo. 

\begin{figure}[ht]
	\centering
	\includegraphics[scale=0.40]{Onset}
	\caption{The onset of a note is the instant which marks start of the variation in the frequency of a signal (Image Source: \cite{onset-tut})}
	\label{fig: Onset}
\end{figure}

Since Longuet-Higgins' first work the area of beat detection has expanded rapidly, in 2005 the first annual Music Information Retrieval Evaluation eXchange (MIREX) was held in 2005. MIREX includes a contest with the goal of comparing state-of-the-art algorithms for music information retrieval \cite{mirex-main}. The topics to be evaluated are proposed by the participants. In the first year, three of the nine topics concerned beat detection, including audio onset detection and since it's first inclusion it has been an evaluated topic of all but one of the last twelve contests \cite{mirex-onset}. The beat detection algorithms proposed to perform the tempo analysis for this project, the Beatroot system developed by Simon Dixon \cite{dixon1} and a development on the original audio analysis using the Discrete Wavelet Transform (DWT) by Tzanetakis, Essel and Cook \cite{tzane1}. Are both former entrants to the MIREX contest, with the Beatroot system receiving the highest score in the 2006 Audio Beat Tracking task \cite{mirex-06} 

\subsection{Project Aims}
The primary aim of this project is to investigation whether some of the currently available beat detection algorithms are accurate enough to form the basis for a training tool to be used by drummers practicing on an acoustic drum kit. In order to achieve this the developed software package, hereafter referred to as  RTT\_Analyser\footnote{Where RTT stands for Real Time Tempo}, will need to process enough live audio in form of sampled drum beats and record each of the chosen beat detection algorithms accuracy. 

The original core features of the RTT\_Analyser developed for this project are:

\begin{enumerate}
\item A live audio tempo analysis tool, that compares and records the performance of selected beat detection algorithms
\item The RTT\_Analyser implements an adapted version of the beat tracking system Beatroot and Discrete Wavelet Transform algorithm described by Tzanetakis \textit{et al}'s \cite{tzane1} which process live audio as opposed to originally designed off-line audio files.
\item RTT\_Analyser implements a system capable of parallel processing, allowing for the same captured live audio data to be sent to the chosen beat detection algorithms in order for the tempo to be calculated simultaneously.
\item While processing live audio the RTT\_Analyser stores a predetermined data set in order to allow for performance analysis of the beat detection algorithms.
\item The RTT\_Analyser will provide the user with real time feedback of the most recent tempo calculation returned
\item An extensive sample set of drum beats will need to be created to ensure the system is tested sufficiently
\end{enumerate}



\maketitle{}\section{RTT\_Analyser Beat Detection Algorithms}
The original proposed solution incorporated two beat detection algorithms, Beatroot system \cite{dixon1} and the DWT method \cite{tzane1}. However, the adaptation of the Beatroot system to be used with live audio took longer than the proposed time-frame. This meant that an alternative system needed to be found in order to mitigate this issue, conveniently the Beatroot software package also contained another beat detection system, the Performance Worm \cite{dixon3}. In order to ensure the project remained on track it was decided to substitute the Performance Worm for the Beatroot system. The intention was, if time allowed, to resume the adaptation of the Beatroot system to work with live audio once the project was ahead of schedule and this was successfully completed prior to the development of the user interface.

The project report will now discuss the beat detection algorithms incorporated in the RTT\_Analyser as well as the system used to allow for simultaneous tempo calculation.  

\subsection{Beatroot}
Beatroot is a beat detection software package created by Simon Dixon \cite{dixon1}, which was designed to extract musical expression information musical recordings \cite{dixon3}. Beatroot was included in the RTT{\_}Analyser as the algorithm designed by Dixon was considered to be sufficiently fast enough to be implemented as part of a real-time system \cite{dixon4}.\par

Beatroot works by first obtaining a time-frequency representation of the signal based on a Short Time Fourier Transform (STFT) using a Hamming window \cite{dixon2}. The STFT is a form of Fourier transform (FT), which in the form of the Discrete Fourier Transform (DFT) can be used to find out how much of each frequency exists in a signal. Th A negative of the DFT is that it is unable to provide any details of when a frequency component occurs in time for non-stationary signals\footnote{Non-stationary signals are signals whose frequency contents changes over time \cite{polikapt2}}. A solution to this is to split a non-stationary signal into a number of smaller segments using a window function, which effectively created a series of stationary\footnote{The frequency contents of a stationary signal does not change over time} signals which the DFT can then be applied to. By splitting the signal into smaller segments the STFT is able to apply the DFT to these segments and essentially express a signal as a linear combination of elementary signals that are easily manipulated. The DFT returns a spectrum that contains information about how the energy held within the signal is distributed in the time and frequency domains \cite{tzane2}. The use of a Hamming window function in the STFT employed by Beatroot is an attempt to reduce the amount of additional frequencies appearing in the returned DFT spectrum, known as spectral leakage, which is caused when the steep sloped of a rectangular window (Figure~\ref{fig: rectan}) causes the frequencies to become distorted. The Hamming window counteracts this by using a bell shape, which has the effect of reducing amplitudes of its sidelobes \cite{lyons} and ensures the spectrum returned is less spread out and closer to the ideal theoretical result \cite{tzane2}. A visualisation of a Hamming window can be seen in Figure~\ref{fig: hamWin}. 


\begin{figure}[ht]
\centering
\begin{subfigure}{.5\textwidth}
  \centering
  \includegraphics[width=0.6\linewidth]{images/rectangWin.jpg}
  \caption{Rectangular window}
  \label{fig: rectan}
\end{subfigure}%
\begin{subfigure}{.5\textwidth}
  \centering
  \includegraphics[width=0.6\linewidth]{images/hamming.jpg}
  \caption{Hamming window}
  \label{fig: hamWin}
\end{subfigure}
\caption{(Image Source: \cite{lyons})}
\label{fig: Wins}
\end{figure}


Once the time-frequency representation is returned the next stage is to apply the spectral flux onset detection function, which is a method for measuring the change in magnitude of each returned frequency bin \cite{dixon2}. The onsets are then selected from the spectral flux onset detection function by a peak-picking algorithm which finds local maxima within the detection function. The next stage is to apply the tempo induction algorithm which is used to compute clusters of inter-onset intervals (IOI) by using the calculated onset times. Each cluster represents a hypothetical tempo, in seconds per beat \cite{dixon1}. The clustering algorithm works by assigning an IOI to a cluster if its difference from the cluster is less than 25ms. The cluster information is then combined by recognising the approximate integer relationships between clusters. An example of this can be seen in Figure~\ref{fig: br-clusters} where cluster C2 is twice as long as C1 and C4 is twice that of C2. This information along with the number of IOIs within a cluster is then used to weight each cluster which is then returned as a ranked list of tempo hypotheses \cite{dixon4}.

\begin{figure}[ht]
	\centering
	\includegraphics[scale=0.25]{br-clusters}
	\caption{(Image Source: \cite{dixon4})}
	\label{fig: br-clusters}
\end{figure}

The multiple agent architecture of Beatroot's beat tracking subsystem is then employed to find the sequences of events that closest match the original tempo hypotheses, each of these sequences is then rated and the most likely set of beat times is determined. Each of the agents is initialised with a tempo or beat rate hypothesis and an onset time. Further beats are then predicted by the agent based on these parameters. Any onsets corresponding with the predicted beat times are taken as a beat time, those falling outside are not considered to be a beat time, although the possibility that the onset is not on the beat is considered. The agents then rate themselves based on an evaluation function which looks at how evenly spaced the beat times are, the number of predicted beats which relate to actual events, and the salience\footnote{The salience is a measure of the note duration, density, pitch and density \cite{dixon1}, which is calculated from the spectral flux of the onset \cite{dixon4}} of the matched onsets. The agent with the highest score is then returned as the sequence of beats corresponding to the processed audio \cite{dixon4}, which is then used to determine the overall tempo of the audio by using the ``inter-beat intervals, measured in beats per second'' \cite{dixon1} to calculate beats per minute (bpm) of the audio.


\subsection{Discrete Wavelet Transform and Beat Detection Method}
The first literature regarding the wavelet was provided by the mathematician Albert Haar in 1909 [18]. The wavelet transform is a technique for analysing signals which was developed as an alternative to the STFT \cite{tzane1}. Like the STFT, the DWT is able to provide time and frequency information, however, unlike the STFT the DWT is able to do this without the need for a window function. The DWT can essentially be considered to be a filter-bank, where a filter-bank is a system used to separate sub-bands by using an array or bank of filters. Where each of the filters corresponds to half frequency range of the closest centre higher frequency. Thus each filter will have half or twice the bandwidth of any of its adjacent filters. 

In 2001, Tzanetakis \textit{et al} described how the Discrete Wavelet Transform (DWT) could be used to extract information from non-speech audio \cite{tzane1}. Their beat detection algorithm was based on the detecting the most prominent signals which are repeated over a period of time within the analysed audio. The first stage is to split the signal into a number of octave\footnote{Acoustically, an octave above a note is the note which has twice the frequency of the original \cite{oxford-comp}.} frequency bands with the DWT. This allows for the time domain amplitude envelopes of each frequency band to be extracted separately. The extraction of these envelopes is completed in the following three steps:

\begin{enumerate}
\item Full Wave Rectification - process of converting the amplitude of each frequency band to one polarity \cite{pallas}, which can be either positive or negative. A visual representation can be seen in Figure~\ref{fig: fwr}.
\item Low Pass Filtering - Low pass filtering is a signal processing technique which is designed to allow frequencies below a cutoff frequency through but blocks any frequencies above the cutoff frequency \cite{smith}.
\item Down-sampling - Due to the large periodicities that can occur in beat analysis, down-sampling the signal reduces the computation time of the autocorrelation stage without causing any negative effects on the performance or the algorithm \cite{tzane3}
\end{enumerate}

\begin{figure}[ht]
	\centering
	\includegraphics[scale=0.25]{FWR2}
	\caption{Visual representation of Full Wave Rectification (diagram adapted from https://en.wikipedia.org/wiki/Rectifier)}
	\label{fig: fwr}
\end{figure}


After these steps each frequency band is normalised through a method of mean removal in order to ensure the signal is centred at zero for the autocorrelation stage. The autocorrelation function is then applied to each frequency band and its peaks correspond to the various periodicities of that signal's envelope. The first five peaks of the function and their corresponding periodicities are then calculated in beats per minute and added to a histogram, this process is repeated while iterating over the signal. The estimated tempo of the audio signal is then retrieved from the periodicity that corresponds to the highest peak within the histogram \cite{tzane1}. The DWT beat histogram flow diagram can be seen in Figure~\ref{fig: dwtFlow}.

\begin{figure}[ht]
	\centering
	\includegraphics[scale=0.35]{images/dwtflow.jpg}
	\caption{Flow diagram of the beat histogram method described by Tzanetakis \textit{et al} \cite{tzane3}}
	\label{fig: dwtFlow}
\end{figure}


Originally, it was intended that Discrete Wavelet Transform beat detection component of the RTT\_Analyser would be implemented by the author. It was recognised in the project proposal that this could result in the rest of the project being delayed and with the alternative solution being to adapt the Matlab implementation of the Tzanetakis \textit{et al} \cite{tzane1} algorithm by Eng Eder de Souza \cite{desouze}. Due to the time spent in investigating how to adapt the Beatroot system to work with live audio a different solution was needed, this was provided by the Scala implementation by Marco Ziccardi \cite{marcoZin}.

\subsection{Performance Worm}
The final beat detection algorithm employed in the RTT\_Analyser is the Performance Worm (PW) system, designed by Simon Dixon, Werner Goebl and Gerhard Widmer \cite{dixonGoeblWidmer}. The PW is based on a real time algorithm which is able to determine the tempo of the input raw audio, while keeping track of other possible tempo hypotheses that are rated and updated dynamically. Allowing for the most recently highest ranked tempo hypothesis to be returned to the user \cite{dixonGoeblWidmer}. 

First the PW processes the raw audio which can either be extracted from a static recording or directly from a live input with a smoothing filter\footnote{Process which attempts to reduce any noise found within a signal. A smoothing filter does this by reducing any points in the signal that are higher than adjacent points and reducing those points that are lower than adjacent point \cite{pragmaticSig}} in order to obtain the RMS amplitude\footnote{RMS stands for root mean square and is the process of squaring all of the amplitude values, then taking the average (mean) of the squared values, and then calculating the square root of the average value \cite{rms}.} of the signal taken from a 40ms window. The note onsets are then calculated by the event detection module that finds the slope of the smoothed amplitude and then calculates the set of local peaks which are taken as the note onset times \cite{dixonGoeblWidmer}. 

The signal is then processed by the multiple tempo tracking subsystem which uses a similar approach to the Beatroot system. The time intervals (inter-onset intervals or IOIs) of event pairs are first calculated. A clustering algorithm (Figure~\ref{fig: clusterAl}) is then applied in order to determine the significant clusters of IOIs, which are subsequently assumed to be the musical units held within the signal. As in the Beatroot system, these clusters are then used as the bases of the tempo hypotheses produced by the tempo tracking subsystem. While running, the clustering algorithm keeps 5 seconds of onset times within memory and begins processing by determining all IOIs between the onsets in memory. The tempo inducer then exploits a property of most Western music where time intervals are related by small approximate integer ratios. As the times held by the clusters can be considered to represent related notes, e.g. quarter and eighth notes. The tempo inducer then adjusts cluster times and weightings according to the information held by the sets of related clusters. Two clusters are considered to be related if the ratio of their time intervals is close to an integer. The highest weighted clusters and their respective tempo hypothesis is then returned as the tempo output \cite{dixonGoeblWidmer}.


\begin{figure}
	\hspace{10mm} For each new onset\\
	\hspace*{20mm} For IOI times $\mathit{t}$ from 100ms to 2500ms in 10ms steps\\
	\hspace*{30mm} Find pairs of onsets which are $\mathit{t}$ apart\\
	\hspace*{30mm} Sum the mean amplitude of these onset pairs\\
	\hspace*{20mm} Loop until all IOI time points are used\\
	\hspace*{30mm} For times $\mathit{t}$ from 100ms to 2500ms in 10ms steps\\
	\hspace*{40mm} Calculate window size $\mathit{s}$ as function of $\mathit{t}$\\
	\hspace*{40mm} Find average amplitude of IOIs in window $\mathit{[t, t + s]}$\\
	\hspace*{40mm} Store $\mathit{t}$ which gives maximum average amplitude\\
	\hspace*{30mm} Create a cluster containing the stored maximum window\\
	\hspace*{30mm} Mark the IOIs in the cluster as used\\
	\hspace*{20mm} For each cluster\\
	\hspace*{30mm} Find related clusters (multiples or divisors)\\
	\hspace*{30mm} Combine related clusters using weighted average\\
	\hspace*{20mm} Match combined clusters to tempo hypotheses and update\\
	\caption{Clustering algorithm used in the Performance Worm Multiple Tempo Tracking Subsystem \cite{dixonGoeblWidmer}}
	\label{fig: clusterAl}
\end{figure}

\maketitle{} \section{Solution Design and Architecture}
The basic premise for the RTT\_Analyser is to enable the user to play a live drum beat through the system and the tempo of the live audio is returned to user as well as being stored for future analysis. Initially, the RTT\_Analyser opens the inbuilt microphone of the device upon which the software is being run. After establishing the live audio stream the RTT\_Analyser beat detection algorithms are sent the live audio data in the format of a byte array. These live audio bytes are then decoded according to the individual algorithm's requirements before being processed and the tempo calculated. A general schematic of the work-flow of the RTT\_Analyser can be seen in Figure (add in).

\subsection{Live Audio Processing}
\label{sec: liveaudio}
The live audio is processed using the Java Sound API, which contains two key types; mixers and lines. Mixers are a representation of the audio devices available on a certain system and a line is an element of the digital audio pipeline responsible for moving audio in and out of the system. Typically, audio capture is started or stopped using a TargetDataLine that provides a path to the systems audio input device, access to this device is provided by the AudioSystem class, which acts a clearing house for audio components. In order to ensure the bytes of audio data passed through the TargetDataLine are interpreted correctly the AudioFormat object enables the audio format to be defined with by the list attributes below \cite{javasound}, which includes the values chosen to ensure that the audio capture by the RTT\_Analyser is comparable to CD quality:

\begin{itemize}
\item Encoding - This will be set to ``PCM.signed'', representing audio encoded to the native linear pulse code modulation, where pulse code modulation is the process of sampling and quantising the signal into discrete symbols for transmissions \cite{pulseWag}.
\item Sample Rate - 44,100, set to match CD quality for the number of analog samples which will be analysed per second. 
\item Sample Size in Bits - 24, based on a sound card with a 24 bit sample depth.
\item Channels - 2, audio will be captured using the built in microphone, typically stereo of the device the RTT\_Analyser is being run on.
\item Frame Size - 6, where the frame size is the number of bytes in a sample multiplied by the number of channels [17].
\item Frame Rate - 44,100, same as sample rate.
\item Big Endian (boolean) - false, as the project will be developed on an Intel core which uses a little-endian architecture\footnote{Endianess refers to the order of bytes which make up a digital word. Big endianess stores the most significant byte at a certain memory address and the remaining bytes being stored in the following higher memory addresses. The little-endian formate reverses the order storing the least significant at the lowest and most significant at the highest memory address \cite{endiness}.}.
\end{itemize}

In order to process the captured bytes the RTT\_Analyser then required the use of a concurrent system to allow for each of the three beat detection algorithms to be run in parallel, which was provided by the Akka Actors system.

\subsection{Akka Actors}
The Akka Actor Model is specifically designed to provide the ability to write concurrent systems with a high level of abstraction that alleviates the developer from the need to handle locking and individual thread management. Therefore, making it much easier to write concurrent and parallel systems when compared to the traditional approaches used in Java \cite{akkaActors}.

A fundamental construct of the Akka Actor system is that it strictly adheres too the Reactive Manifesto, which aims to ensure applications built under this are easier to develop and are amenable to change, allowing for a higher level of tolerance to failure and the ability to meet any such failure elegantly as opposed to disaster \cite{reactMan}. The Reactive Manifesto requires applications to satisfy one or more of the requirements listed below \cite{reactMan} and visualisation of how they interact can be seen in Figure~\ref{fig: reacMan}.


\textbf{Responsive} - The system responds in a timely manner, if possible. By providing usability and utility responsiveness allows for any problems to be detected and resolved effectively.\\\\
\textbf{Resilient} - The system should be able to remain responsive even in the event of a failure, which applies to all parts of the system not just the mission critical components. To achieve this each component needs to be isolated within the system, therefore allowing failed parts of the system to recover without effecting the responsiveness of the system as a whole.\\\\
\textbf{Elastic} - The system will be able to maintain the same level of responsiveness despite varying workloads, reacting to changes in the rate of input by adapting the levels of resources allocated to service these inputs accordingly.\\\\
\textbf{Message Driven} - Reactive systems rely on the use of asynchronous message passing in order to establish boundaries between components within the system, ensuring isolation, loose coupling\footnote{Loose coupling is a design approach of distributed systems which emphasises agility and ability to adapt to changes \cite{looseCouple}} and location transparency\footnote{The decoupling of the runtime instances from their references \cite{reactMan}}. By utilising a non-blocking\footnote{In concurrent programming algorithms which do not have their execution indefinitely postponed when competing for a resource is said to be non-blocking \cite{reactMan}.} message-passing it is possible to ensure message recipients only consume resources when active, leading to a much reduced system overhead.\\\\

\begin{figure}[ht]
	\centering
	\includegraphics[scale=0.25]{images/reacManif.jpg}
	\caption{Visualisation of the requirements of the Reactive Manifesto interact \cite{reactMan}}
	\label{fig: reacMan}
\end{figure}


\subsubsection{The Actors}
The actors within the actor model are objects which are used to encapsulate state and behaviour. They communicate exclusively through messages passed to a recipients mailbox and it can help to think of them as group of people being assigned subtasks within an organisational structure. One of the key features of an actor system is that tasks are split up and delegated until they become small enough to be handled in one piece. This process not only ensures that the tasks being carried out by the actors are clearly defined but also the actors themselves can be designed in terms of the messages they are able to receive and how they should react to these messages and any failures that might occur. The actors within a system will naturally form a set hierarchy. For example, an actor system where an actor assigned the task of overseeing a certain function might split this function into a number of smaller tasks which it assigns to some child actors that it supervises until the task is complete \cite{acotrsys}. The actors created within the RTT\_Analyser were a mix of stand alone actors and actors which were required to supervise subtasks carried out by child actors.

\subsection{Design Pattern}
The RTT\_Analyser was developed using the Model View Controller (MVC) design pattern, as the MVC pattern is the tried-and-tested approach when designing applications with a graphical user interface \cite{designPatterns}. The MVC pattern is made up of the following components and a visualisation of the MVC can be seen in Figure~\ref{fig: mvc}. 

\begin{itemize}
\item \textbf{Model} - is represented by the data and associated applications \cite{designPatterns}, within the RTT\_Analyser the model is provided by the beat detection algorithms, the live audio capturing and processing components.\\
\item \textbf{View} - is the graphical interface displayed to the user which updates itself automatically \cite{designPatterns}. For the RTT\_Analyser the view is represented by the user interface\\
\item \textbf{Controller} - is the part of the application that responds to the user interactions, liasing with both the model and the view components\cite{designPatterns}. The controller for the RTT\_Analyser is the actor system, which is responsible for message passing between the model and view components.
\end{itemize}

\begin{figure}[ht]
	\centering
	\includegraphics[scale=0.25]{images/mvc.png}
	\caption{The three components which make up Model View Controller design pattern \cite{applemvc}}
	\label{fig: mvc}
\end{figure}

A high level class diagram of the final version of RTT\_Analyser can be seen in Figure~\ref{fig: uml} 

\begin{figure}
	\centering
	\caption{Image to be added}
	\label{fig: uml}
\end{figure}


\maketitle{} \section{Implementation}
The RTT\_Analyser is written in Java and Scala, due to the Beatroot and Performance Worm both being written in Java, and the author's familiarity with both languages. Using Scala also ensured the twin constructs of pattern matching and case classes were able to be utilised where appropriate. Pattern matching is used to implement type switches and is considered to create ``code that is both succinct and obviously correct'' \cite{mariusEr}. Case classes are Scala's way of allowing pattern matching to be performed on objects without the need for additional boilerplate code. Using Scala, also allows for more elegant and concise tail recursive\footnote{A recursive solution is considered to be tail recursive if the last operation carried out is to calculate the return value, it has the advantage over recursive calls of not building a stack trace and instead performing each operation in a single stack frame \cite{odesky}} solutions to be developed over loop-based solutions. As the Scala compiler provides an optimiser which ensures that there are not any runtime overheads to be paid for employing tail recursion \cite{odesky}.

The RTT\_Analyser was developed in the Scala specific build tool, sbt (simple build too), a build tool that creates a stable build platform increasing productivity by utilising some good ideas from other build tools like, minimal configuration and built-in tasks including test, compile and publish. It is also able to support a reactive development environment that is able to re-run all tests when source code is updated \cite{sbt}. In terms of this project sbt was chosen as it supports mixed Scala/Java projects. Allowing for the Beatroot and Performance Worm systems written in Java to be easily incorporated with the Scala implementation of the Discrete Wavelet Transform and the Akka Actor system.

\subsection{The Actor System}
Although the actors were not the first element of the RTT\_Analysert to be implemented, introducing them first helps to provide some context for the how the other components communicate. The actor system is constructed of five different actors in total and the messages permitted to be passed between the RTT\_Analyser's actors were all implemented as case classes within a sealed trait (Figure~\ref{fig: messages}). Using a sealed trait ensures that the only classes extending the trait are in the same file which is very useful for pattern matching\cite{odesky}. Therefore, working well with the actor's recieve method which works on pattern matching.\par

\begin{figure}
\centering
\includegraphics[scale=0.25]{images/messages.jpg}
\caption{The message case classes used by the RTT\_Analyser's actor system}
\label{fig: messages}
\end{figure} 

The LiveAudioActor (Figure~\ref{fig: liveaudioactor}) assumes the role of the overseer  of the system, being responsible for starting and stopping the live audio capture, as well as the termination of the RTT\_Analyser when appropriate. \par

\begin{figure}
\centering
\includegraphics[scale=0.3]{images/LiveAudioActor.jpg}
\caption{LiveAudioActor class}
\label{fig: liveaudioactor}
\end{figure}

The ProcessingActor (Figure~\ref{fig: processingActor}), a child of the LiveAudioActor can be considered the coordinator of the storing and processing captured audio data, and the writing of the calculated tempos. In order to ensure the processing remains as isolated as possible the ProcessingActor employs two worker actors, BeatrootActor and DWTActor, carrying out the two tasks of sending the captured data and tempo calculation. The beat detection worker actors do not return any messages to the ProcessingActor, instead the immutable ActorRef handle which allows for the current instance of the respective actor is passed the respective beat detection system. Therefore, the calculated tempo values are then sent back directly to the ProcessingActor to be stored and subsequently saved to file when appropriate.\par

\begin{figure}
\centering
\includegraphics[scale=0.3]{images/processingRec.jpg}
\caption{Recieve method of the ProcessingActor class}
\label{fig: processingActor}
\end{figure}

The final RTT\_Analyser actor was a standalone actor on the same level as the LiveAudioActor within the actor hierarchy which was responsible for the timing of the beat detection test.\par

All of the RTT\_Analyser's parent and child actors were all instantiated in the self contained Operator object (Figure~\ref{fig: props}. This ensured that potentially dangerous practice of declaring one actor and breaking actor encapsulation \cite{akkaActors} was avoided.  

\begin{figure}
\centering
\includegraphics[scale=0.35]{images/actorprops.jpg}
\caption{Declaring of the RTT\_Analyser's actors}
\label{fig: props}
\end{figure}

\subsection{Live Audio Capture}
To implement the RTT\_Analyser's live audio capture with the Java Sound API initially, the SoundCaptureImpl class was created in Java and then converted to Scala to be responsible for the live audio capture and holding of the raw audio in a ByteArrayOutputStream before being passed for processing by the beat detection algorithms. The sound capturing components of this class were based on those described in the Java Sound Programmer Guide \cite{javasound} and consisted of the code in Figure~\ref{fig: startcap}.

\begin{figure}[ht]
\centering
\includegraphics[scale=0.3]{images/startCap.jpg}
\caption{Example of the basis of live audio capture method in Scala}
\label{fig: startcap}
\end{figure}

The audio format described in section ~\ref{sec: liveaudio} is then applied to the TargetDataLine, which provides access to the built in microphone of the device, if available. The TargetDataLine class is then attached to an instance of the Java AudioInputStream class which converts the captured audio into a stream of bytes for processing \cite{soundTrail}. 

The need for the code in Figure ~\ref{fig: startcap} however became defunct with the design decision to include the Performance Worm system within the RTT\_Analyser as the Performance Worm was already set up to capture live audio. Therefore the captured audio now needed to be stored by the Performance Worm in a manner which allowed for all three beat detection algorithms to be able to process the same captured audio. This was achieved by adding the method addBytes (Figure~\ref{fig: addBytes}) to the AudioWorm class, responsible for the capturing and processing of the live audio within the Performance Worm. The call to the addBytes was added just below the \textit{.read} call of the AudioInputStream, the byte array was then passed to the addBytes method and stored within a ByteArrayOutputStream to be processed later. Once the sufficient number of bytes had been amassed the ProcessingActor was sent the ProcessBytes message using the Java \textit{.tell} method which initiated the processing of the audio data by the Beatroot and DWT beat detection systems.

\begin{figure}[ht]
	\centering
	\includegraphics[scale=0.25]{images/addBytes.jpg}
	\caption{}
	\label{fig: addBytes}
\end{figure}

\subsection{DWT Beat Detection Implementation}
The implementation of Tzanetakis \textit{et al} beat detection algorithm by Marco Ziccardi \cite{marcoZin} was part of a larger system which offered a number of other beat detection methods. The RTT\_Analyser only required the class which implemented the DWT beat detection algorithm, WaveletBPMDetector, which originally relied on a Scala version of the WavFile Java class created by Andrew Greensted \cite{green}, however as the RTT\_Analyser was not required to decode WAV files, this class was adapted to form the LiveAudioProcessor class (Figure~\ref{fig: liveAudPro}). The class was based on three methods from Andrew Greensted's WavFile class \cite{green}, readFrames, an overloaded readFrames and the getSample method. Rather than being simply a Scala version of the Java code, where possible these methods were written using Scala's pattern matching facility and any loops were converted into a tail recursive solution.\par

The final addition to the LiveAudioProcessor class was the inclusion of the addData method that takes a byte array containing the raw live audio bytes to be processed and assigns them to the buffer variable which is processed in the getSample method. In order for the WaveletBPMDetector class to return a tempo result for a single window of audio (matched to the size of the buffer byte array), the for loop was removed from the bpm method. Once a tempo was calculated this was sent back to the ProcessingActor with the New

\begin{figure}[ht]
	\centering
	\includegraphics[scale=0.25]{images/liveAudioPro.jpg}
	\caption{LiveAudioProcessor}
	\label{fig: liveAudPro}
\end{figure}


\subsection{Tempo, Analyser and Stats}
The Tempo object was implemented as a method of encapsulating the RTT\_Analyser's calculated tempo results. Comprising of the calculated and expected tempos, and difference between these values, an Option\footnote{The Option type is Scala's alternative to Java's null value, the Option type is in fact a container which either holds a value or is empty. It provides a much safer alternative to the error prone null value \cite{mariusEr}.} count of beats detected and the time at which the tempo result was calculated.\par 

The Analyser object was designed to hold the calculated Tempo objects within a ListBuffer, chosen for its scalability. The Analyser trait was extended to create three case classes corresponding to the relevant beat detection algorithm. The Analyser implementations also contained an Option Stats value. The Stats object, a case class used to hold the average and median values of the calculated tempos, the difference between the expected and calculated tempo, and the time taken to calculate a tempo within plus or minus one bpm of the expected value. Useful for assessing the efficiency of the beat detection algorithms. The StatsCalculator case class was a helper class which was responsible for carrying out computation of the values held with the Stats object and was implemented using the code seen in Figure~\ref{fig: statsC}.

\begin{figure}[htbp]
\centering
\includegraphics[scale=0.35]{images/statscalc.jpg}
\caption{Stats calculation methods used by the StatsCalculator}
\label{fig: statsC}
\end{figure}

\subsection{JSON Parsing}

The implementation of the data storage system of the RTT\_Analyser used the JavaScript Object Notation (JSON) as the data generated was considered not large enough to warrant the creation and management of an SQL database. JSON's lightweight data-interchange format \cite{json} ensured the conversion from Scala case class to JSON object is straightforward. The JSON is generated the ScalaJSON library provided by the play framework \cite{play}, specifically the \textit{Writes} object which converts an object into a JSON representation held encapsulated within a \textit{JsValue} object. Three methods were written in order to account for the three implementations of the Analyser trait, an example of the code can be seen in Figure(add one). As JSON is written in a text format \cite{json}, the parsing of generated JSON was performed by simpling converting the JSON objects generated by the respective write methods (example of the write method can be seen in Figure~\ref{fig: json}) to a string which was then written to a JSON file using the flush method in Figure{add one}, where the path was provided from the user interaction with the RTT\_Analyser's user interface.

\begin{figure}[htbp]
\centering
\includegraphics[scale=0.25]{images/writes.jpg}
\caption{Example of the write method used to convert the Tempo and Stats object values to JSON}
\label{fig: json}
\end{figure}

\subsection{Adaption of Beatroot}
Once the development of the RTT\_Analyser was considered to be sufficiently on schedule the design decision was taken to attempt the adaption of the Beatroot system to work with live audio again. Much like with the WaveletBPMDetector, Beatroot would have to be set up to use the same audio bytes as those captured by the Performance Worm, to ensure all systems were using the exact same captured audio. As the AudioWorm was already set up to capture the enough bytes to work with the WaveletBPMDetector's smallest working window size of 131072, the same figure was chosen to be processed by Beatroot. \par
With the correct design decisions now taken the live audio adaption of Beatroot consisted of only a few minor steps. First, the processFile method was amended to receive the byte array holding the audio data recorded by the Performance Worm. A ByteArrayInputStream was then instantiated to hold the data, allowing the processFrame and getFrame methods to work in the same manner as with a upload static audio file. After the captured audio was processed the ByteArrayInputStream was closed, a step necessary to ensure that no overlap existed between captured audio data sets sent to be processed.

\subsection{RTT\_Analyser User Interface and HTML Results Viewer}
The RTT\_Analyser's user interface (Figure~\ref{fig: rtt} was designed with simplicity in mind and was implemented using the ScalaFX, which sits on the top of the JavaFX API and uses the same declaration syntax as normal objects within Scala. Enabling developers to use the same operators and syntax to create and modify the scene graph\footnote{A hierarchy of tree nodes what represent the visual components of an application's user interface\cite{oracle2}} \cite{scalafx}. Before starting the user could begin live audio capture the expected bpm, file name and path were required. Once input, there are two modes available to the user. The first is to control the live audio beat detection manually using the start and stop buttons accordingly. The second, is the test mode which will runs the audio capture and beat detection for thirty seconds, before closing down audio inputs and storing the results. During execution in both modes the RTT\_Analyser displays the calculated tempo values to the user as a bpm value rounded to two decimal places. \par

\begin{figure}[htbp]
\centering
\includegraphics[scale=0.25]{images/rtt.jpg}
\caption{RTT\_Analyser's User Interface}
\label{fig: rtt}
\end{figure}

On completion within both modes the results are stored within a JSON file automatically and also written to an HTML file so the results can be viewed in a web browser using the \textit{Results} button. The HTML tables were produced HtmlWriter object which used reflection where possible to obtain the relevant case class constructor field names, before using a number of tail recursive methods to write the field values and HTML tags to a StringBuilder, chosen for it's mutability. The HTML tables are subsequently saved to an HTML file which is opened and viewed via the default browser using the \textit{Results} button. An example of the displayed results can be seen in Figure~\ref{fig: htmlView}.

\begin{figure}[htbp]
\centering
\includegraphics[scale=0.3]{images/htmlResults.jpg}
\caption{Example of the RTT\_Analyser's HTML results view}
\label{fig: htmlView}
\end{figure}

\maketitle{}\section{Software Testing}
During the development of the RTT\_Analyser, Test Driven Development (TDD) was employed where possible. TDD is a development process which employs test-first development and refactoring. Test-first development involves writing a test before just enough production code is written to pass that test \cite{tdd1}. Refactoring is a process of making small changes to code, is used to improve the design of the code, thus making it easier to understand and modify \cite{tdd2}. \par

The tests were written using the ScalaTest framework, which is one of the most dominant testing frameworks currently available. ScalaTest is designed with the intention of providing a test framework that makes testing more concise and uses the Scala language in a way to ensure testing is easy and fun \cite{testingScala}.

TDD was used at any stage of the development process where a piece of code with an expected behaviour was being implemented, it was not used for all classes as when dealing with live audio it was not possible to produce a full suite of tests for all of the functionality. Although, where the requirement was simply to return a value of a certain type, basic TDD tests were written. The classes that were ideally suited to TDD and were developed completely using it were the Tempo, Analyser implementations, StatsCalculator, JSONParser and HTMLWriter. \par

A good example of TDD being applied during the development of the RTT\_Analyser was during the creation of the StatsCalculator. Designed to carry out average and median calculations on the stored Tempo objects held within the Analyser implementations enabled TDD to be employed successfully.

When it was not possible to use TDD, particularly during the initial set-up and testing of the beat detection algorithms with live audio. A simpler testing approach was taken, consisting of running the system with a selection of sample drum beats of varying tempos, with console logs used to test that the system was performing the desired tempo calculations.

\maketitle{}\section{RTT\_Analyser Beat Detection Testing}
\subsection{Drum Beat Sample Set}
Once the RTT\_Analyser was working with the initial test sample drum beats the full testing cycle could begin. In order to carry out this testing a set of drum beats were written using the Apple sequencer program, GarageBand \cite{garage}. The writing of the drum beats took a slight deviation from the project proposal in that rather than create multiple full drum beats in a varying number of styles, the drum beats were instead tested in layers. For example, a simple four four (4/4) bass drum beat with a back beat and the time element being played on the hi-hat cymbal would account for three samples. The reason for this decision stemmed from how a cymbal behaves when it is played at high amplitudes (loudness), the vibrations it create become chaotic and its clearly identifiable signals disappear and it effectively becomes noise \cite{soundonsound}. Therefore in order to compare the drum sample with cymbals a baseline was required. Additionally, this change to the project proposal was considered to be able to provide a much more diverse data set which would hopefully help the understanding of how the different elements of a drum kit, not just cymbals, can affect the accuracy of the beat detection algorithms being tested by the RTT\_Analyser.\par

In total number twenty eight drum samples were produced and the full details of the elements which each drum beat was comprised of and style of the drum beat can be seen in Table~\ref{tab: beats}. A full visualisation of the drum beats used can be found in Appendix A.

\subsection{Beat Detection Testing}
The full beat detection tests were carried out in an environment with as little background noise as possible, where the drum beats were played on an external speaker set to the same volume for all tests. Each drum beat was tested for thirty seconds over the tempo range of 60-160 bpm, the tempo was increased in five bpm increments. Resulting in twenty one tests for being carried out for each drum beat and producing a sample set consisting of five hundred and eighty eight (588) results files.

During the testing it became apparent that at lower tempos the RTT\_Analyser was returning values which were roughly twice that of the expected tempo. A possible cause for this was considered to be aliasing, caused when it is not possible to distinguish between the values from sampled audio as they contain periodic replications of the original spectrum\cite{lyons}. To investigate if this was the root cause and as this behaviour was being displayed by all three of the beat detection algorithms. It was decided that all of the drum samples would be reprocessed through the RTT\_Analyser but this time a low-pass filter would applied to the point at which the audio is captured. The low-pass filter was provided by the source\_code.biz Java dsp collection.

\begin{table}[htbp]
\caption{List of Drum Beats Used in RTT\_Analyser Tests} 
\centering
\begin{tabular}{|p{3cm}|p{5cm}|l|}
\hline
\textbf{Drum Beat Code} & \textbf{Elements} & \textbf{Style} \\ [0.5ex]
\hline 
AMPEDUP & All & Heavy rock, GarageBand stock drum beat\\
\hline 
HT & High tom-tom & Straight\\
\hline 
HTLT & High tom-tom and low tom-tom & Straight\\
\hline 
HTLT2 & High tom-tom and low tom-tom & Straight\\
\hline 
K & Bass drum & Straight\\
\hline 
KS & Bass drum and snare drum & Straight\\
\hline 
KS2 & Bass drum and snare drum & Straight\\
\hline 
KSCR & Bass drum, snare drum and crash cymbal & Straight\\
\hline 
KSFTTF & Bass drum (four to the floor) and snare drum & Straight\\
\hline 
KSH & Bass drum, snare drum and hi-hat cymbal & Straight\\
\hline 
KSH2 & Bass drum, snare drum and hi-hat cymbal & Straight\\
\hline 
KSH2CR & Bass drum, snare drum, hi-hat and crash cymbals & Straight\\
\hline 
KSHCR & Bass drum, snare drum, hi-hat and crash cymbals & Straight\\
\hline 
KSHHTLT & Bass drum, snare drum, hi-hat cymbal, high tom-tom and low tom-tom & Straight\\
\hline 
LT & Low tom-tom & Straight\\
\hline 
MTREVIS & All & Mowtown swing, GarageBand stock drum beat\\
\hline 
OFF-KSH & Bass drum, snare drum and hi-hat cymbal & Offbeat played on hi-hat\\
\hline 
OFF-KSHCRFTTF & Bass drum (four to the floor), snare drum, hi-hat and crash cymbals & Offbeat played on hi-hat\\
\hline 
OFF-KSHFTTF & Bass drum (four to the floor), snare drum, hi-hat cymbal & Offbeat played on hi-hat\\
\hline 
OFF-KSHHCRFTTF & Bass drum (four to the floor), snare drum, hi-hat and crash cymbals & Offbeat played on hi-hat\\
\hline 
S & Snare drum & Straight\\
\hline 
SMASH & All & Punk rock (loud), GarageBand stock drum beat\\
\hline 
SW-H & Hi-hat cymbal & Swing\\
\hline 
SW-K & Bass drum & Swing\\
\hline 
SW-KH & Bass drum and hi-hat cymbal & Swing\\
\hline 
SW-KS & Bass drum and snare drum & Swing\\
\hline 
SW-KSH & Bass drum, snare drum and hi-hat cymbal & Swing\\
\hline 
SW-R & Ride cymbal & Swing\\
\hline 
SW-SH & Snare drum and hi-hat cymbal & Swing\\
\hline
\end{tabular}
\label{tab: beats}
\end{table}

\maketitle \section{Results and Analysis}
The twenty eight sample beats processed by the RTT\_Analyser yielded one hundred and thirty seven thousand, one hundred and eighty (137,180) separate tempo records. A breakdown of the number of tempo records recorded per beat detection algorithm can be seen in Figure~\ref{fig: totrec}. The separate JSON files were then consolidated into one file, converted into a CSV file and then connected to the data visualisation software package, Tableau\cite{tableau}.

\begin{figure}
\caption{total records}
\label{fig: totrec}
\end{figure}

\subsection{Overall Average Difference}
The difference between the expected tempo and the calculated tempo was recorded for each tempo record. Calculated by subtracting the returned tempo from the expected tempo~\ref{eq: diff}. Therefore, if the difference is negative then the calculated tempo was greater than what was expected and less than if it is positive.

\begin{equation}\label{eq: diff}
\begin{split}
difference& =expected tempo - calculated tempo\\
\end{split}
\end{equation}

The average difference for each of the three beat detection algorithms beat samples split between style can be seen in Figure~\ref{fig: aveDiff}. The best performing beat detection method can be seen to be the DWT based algorithm by Tzanetakis \textit{et al}'s \cite{tzane1}, for the samples without a filter in a Swing style. As mentioned previously, during the testing process some of the tempo result returned were double that of the expected bpm value, possibly due to aliasing. A solution to the aliasing issue was to add a low-pass filter and looking at the results on the whole it can be considered to have worked to a certain degree. For six out of the nine sets of results those samples processed with a low-pass filter produced a lower average difference than those without a filter. Although, the margin of error between the expected tempo and the calculated tempo are still suitably large.\par

\begin{figure}
\caption{AveDiff}
\label{fig: aveDiff}
\end{figure}

In an attempt to investigate this further a flag was applied to the results set to highlight any tempo result approximately double the expected value. A record was flagged if it the calculated tempo divided by the expected tempo returned a result greater or equal to 1.95~\ref{eq: flag}. The flagged results were then filtered out of the data set and the updated results can be see in Figure~\ref{fig: upAveDiff}. With the flagged results removed the average difference value for all of the beat detection algorithms was greatly reduced except for the swing samples processed by the PW. Suggesting, the PW may be less accurate when processing samples played with a swing style.

\begin{equation}\label{eq: flag}
\begin{split}
Calculated Tempo/Expected Tempo >= 1.95\\
\end{split}
\end{equation}

\begin{figure}
\caption{updatedAveDiffs}
\label{fig: upAveDiff}
\end{figure}

Rather than just removing these results altogether, the next step was to attempt to correct any of the flagged tempo results which when divided by two were within plus or minus five percent of the expected tempo result. With this adaption applied to the data set (seen in Figure~\ref{fig: corAveDiff}), it is clear that overall Tzanetakis \textit{et al}'s \cite{tzane1} DWT method is the most accurate. The different musical styles within the samples also clearly plays a part in their accuracy with the swing samples producing the least accurate results for all three algorithms.


\begin{figure}
\caption{correctedAveDiff}
\label{fig: corAveDiff}
\end{figure}

All of the average difference values used from this point use the corrected values described in the previous section. Versions of the displayed uncorrected and flagged results can be found in Appendix...

\subsection{Single Drum Sample Results}
The next data set consists of any samples which were played using only one drum. All of the samples except for the HT (high tom-tom) and LT (low tom-tom) in the straight sample set were used as parts in the final complete drum patterns.\par

\subsubsection{Straight Single Drum Sample Results}

The straight single drum result set can be seen in Figure~\ref{fig: ssdAveDiff}, again the Tzanetakis \textit{et al}'s \cite{tzane1} DWT method calculated tempos with the highest level of accuracy. Performing the best for three of the four sample sets, although actually returning the lowest level of accuracy of the three beat detection methods the for the hi-hat sample.\par

\begin{figure}
\caption{Straight Single Drum Average Diff}
\label{fig: ssdAveDiff}
\end{figure}

\subsubsection{Swing Single Drum Sample Results}

Unlike the straight single drum data set the results produced for the swing samples were a lot more varied. The Tzanetakis \textit{et al}'s \cite{tzane1} DWT method was again the stand out performer. However, the average difference values returned by the Beatroot and PW methods.

\subsection{Straight Sample Results}

\subsection{Swing Sample Results}

\subsection{Off-Beat Sample Results}

\subsection{Low Tempo Comparison}
During the testing process it was noted that the result doubling seemed to be prominent for samples being played at the lower tempos. Usually anything with a bpm of one hundred and below. To investigate if the algorithms performed better with higher tempos regardless of style, a doubling count was applied to the data set. Using the same five percent margin applied previously. The comparison for tempos was made by visualising the double count and non-double count for the tempos of sixty to eighty, and the expected tempo values equal to twice the low tempo value (one hundred and twenty to one hundred and sixty. 

\begin{figure}
\caption{Low Tempo Comparison}
\label{fig: lowTemCom}
\end{figure}

The results viewed in Figure~\ref{fig: lowTemCom}, confirm that the all three of the beat detection algorithms produce significantly less doubling when processing drum beats with higher tempos. If the style grouping is introduced, it is apparent that the doubling is most prevalent with the straight sample set. Out of the three algorithms, the PW returns the highest count of doubled results suggesting a possible flaw in how the beat inducing algorithm converts peaks in the signal to the returned tempo result.

\subsection{Beat Detection Response Time}
Due to the smaller window size utilised by the PW it is expected to perform the best within this comparison. It is nevertheless an interesting point to investigate. The response time was recorded by the RTT\_Analyser when the first returned tempo was plus or minus one bpm from the expected tempo result. 

Before this could be calculated the all of the tempo results produced by the PW for expected tempos of seventy five were removed. As the PW's first result returned regardless of expected tempo was always seventy five. Despite this and as expected the PW returned the fastest tempo value. Returning a tempo within the margin of error for the straight KSH2 beat within 0.675 seconds when processing through the low-pass filter. Then without a filter a tempo was produced in 0.660 seconds. Demonstrating that if any future drum training tools require instant feedback then the PW needs to be considered.

\maketitle \section{Evaluation and Discussion}

dwt hihat worse

\newpage
\begin{thebibliography}{99}
\bibitem{midi}
\textit{http://www.instructables.com/id/What-is-MIDI/}
\bibitem{oxford-comp}
Alison Latham\\
\textit{The Oxford Companion to Music, 2002, Oxford University Press}
\bibitem{brit-metro} 
\textit{https://www.britannica.com/art/metronome}
\bibitem{drum-bible}
Mick Berry and Jason Gianni\\
\textit{The Drummer's Bible: How to Play Every Drum Style from Afro-Cuban to Zydeco, Second Edition, 2004, See Sharp Press}
\bibitem{drum-note}
\textit{http://www.drummagazine.com/lessons/post/drumkey/}
\bibitem{allen-danneburg}
Allen and Dannenberg\\
\textit{Tracking Musical Beats in Real Time, International Computer Music Conference, International Computer Music Association, 1990, pp. 140-143}
\bibitem{longeut1}
H. C Longuet-Higgins\\
\textit{Perception of melodies, Nature Vol. 263, 1976, pp. 646-653}
\bibitem{dixon1}
Simon Dixon\\
\textit{Automatic Extraction of Tempo and Beat from Expressive Performances. Journal of New Music Research, 30 (1), 2001, pp 39-58}
\bibitem{dixon2}
Simon Dixon\\
\textit{Onset Detection Revisited, Proceedings of the 9th International Conference on Digital Audio Effects, Montreal, September 2006, pp 133-137}
\bibitem{dixon3}
Simon Dixon\\
\textit{On the Analysis of Musical Expression in Audio Signals. Storage and Retrieval for Media Databases, SPIE-IS\&T Electronic Imaging, SPIE Vol. 5021, 2003 pp 122-132}
\bibitem{mirex-onset}
\textit{http://www.music-ir.org/mirex/wiki/2016:Audio\_Onset\_Detection}
\bibitem{sonic}
\textit{http://www.ieor.berkeley.edu/~ieor170/sp15/files/Intro-to\_Sonic\_Events\_Campion.pdf}
\bibitem{onset-tut}
Juan Pablo Bello, Laurent Daudet, Samer Abdallah, Chris Duxbury, Mike Davies, and Mark B. Sandler 
\textit{A Tutorial on Onset Detection in Music Signals, IEEE TRANSACTIONS ON SPEECH AND AUDIO PROCESSING, VOL. 13, NO. 5, 2005, pp. 1035 - 1047}
\bibitem{mirex-main}
\textit{http://www.music-ir.org/mirex/wiki/2005:Main\_Page}
\bibitem{tzane1}
George Tzanetakis, Georg Essl and Perry Cook\\
\textit{Audio Analysis using the Discrete Wavelet Transform, Proc. WSES International Conference on Acoustics and Music: Theory and Applications (AMTA), 2001}
\bibitem{mirex-06}
\textit{http://www.music-ir.org/mirex/wiki/2006:Audio\_Beat\_Tracking\_Results}
\bibitem{dixon4}
Simon Dixon\\
\textit{Evaluation of the Audio Beat Tracking System BeatRoot. Journal of New Music Research, 36, 1, 2007, pp 39-50}
\bibitem{polikapt2}
\textit{http://users.rowan.edu/ polikar/WAVELETS/WTpart2.html}
\bibitem{lyons}
Lyons\\
\textit{Understanding Digital Signal Processing}
\bibitem{tzane2}
Tao Li, Mitsunori Ogihara, George Tzanetakis\\
\textit{Music Data Mining, CRC Press, 2002, pp 45-53}
\bibitem{tzane3}
George Tzanetakis and Perry Cook\\
\textit{Musical Genre Classification of Audio Signals}
\bibitem{smith}
Steven W. Smith\\
\textit{The Scientist and Engineer's Guide to Digital Signal Processing, California Technical Publishing, 2011, Chapter 3}
\bibitem{pallas}
Ramón Pallás-Areny and John G. Webster\\
\textit{Analog Signal Processing, Wiley, 199, pp. 231}
\bibitem{dixonGoeblWidmer}
Simon Dixon, Werner Goebl and Gerhard Widmer\\
\textit{The Performance Worm: Real Time Visualisation of Expression based on Langner’s Tempo-Loudness Animation, International Computer Music Conference, 16 - 21 September 2002, Göteborg, Sweden, pp 361-364.}
\bibitem{pulseWag}
Bill Waggener\\
\textit{Pulse Code Modulation Techniques}
\bibitem{soundTrail}
Oracle\\
\textit{https://docs.oracle.com/javase/tutorial/sound/}
\bibitem{akkaActors}
Akka\\
\textit{http://doc.akka.io/docs/akka/2.4.9/scala/actors.html}
\bibitem{reactMan}
Jonas Bonér, Dave Farley, Roland Kuhn, and Martin Thompson\\
\textit{Reactive Manifesto, http://www.reactivemanifesto.org/}
\bibitem{acotrsys}
Akka\\
\textit{http://doc.akka.io/docs/akka/2.4.9/general/actor-systems.html}
\bibitem{sbt}
Josh Suereth and Matthew Farwell\\
\textit{SBT in Action: The simple Scala build tool, Manning Publications, 2015}
\bibitem{javasound}
Oracle\\
\textit{https://docs.oracle.com/javase/8/docs/technotes/guides/sound/programmer\_guide/contents.html}
\bibitem{desouze}
\textit{https://github.com/ederwander/Beat-Track}
\bibitem{marcoZin}
\textit{https://github.com/mziccard/scala-audio-file}
\bibitem{green}
Andrew Greensted
\textit{http://www.labbookpages.co.uk/audio/javaWavFiles.html}
\bibitem{mariusEr}
Marius Eriksen\\
\textit{Effective Scala, http://twitter.github.io/effectivescala/}
\bibitem{odesky}
Martin Odersky, Lex Spoon and Bill Venners\\
\textit{Programming in Scala, Second Edition, Artima Press, 2010}
\bibitem{play}
\textit{https://www.playframework.com/documentation/2.5.x/ScalaJson}
\bibitem{json}
\textit{http://www.json.org/}
\bibitem{play-writes}
\textit{https://www.playframework.com/documentation/2.5.x/api/scala/index.html\#play.api.libs.json.Writes}
\bibitem{scalafx}
\textit{http://www.scalafx.org/docs/home/}
\bibitem{tdd1}
\textit{http://www.agiledata.org/essays/tdd.html}
\bibitem{tdd2}
\textit{http://www.agiledata.org/essays/databaseRefactoring.html}
\bibitem{testingScala}
Daniel Hinojosa\\
\textit{Testing in Scala, O’Reilly Media, 2013}
\bibitem{garage}
\textit{http://www.apple.com/uk/mac/garageband/}
\bibitem{soundonsound}
\textit{http://www.soundonsound.com/sos/may02/articles/synthsecrets0502.asp}
\bibitem{pragmaticSig}
Tom O'Haver\\
\textit{A Pragmatic Introduction to Signal Processing with applications in scientific measurement, https://terpconnect.umd.edu/~toh/spectrum/, last update 2016}
\bibitem{rms}
\textit{https://csd.wisc.edu/vcd202/rms.html}
\bibitem{looseCouple}
Doug Kaye\\
\textit{Loosely Coupled: The Missing Pieces of Web Services, RDS Press, 2003}
\bibitem{applemvc}
\textit{https://developer.apple.com/library/mac/documentation/General/Conceptual/DevPedia-CocoaCore/MVC.html}
\bibitem{designPatterns}
Tony Bevis
\textit{Java Design Pattern Essentials, Ability First, 2012}
\bibitem{endiness}
\textit{http://www.cs.umd.edu/class/sum2003/cmsc311/Notes/Data/endian.html}
\bibitem{oracle2}
\textit{http://docs.oracle.com/javase/8/javafx/get\-started\-tutorial/jfx\-architecture.htm\#A1106328}
\bibitem{tableau}
\textit{http://www.tableau.com/}
\end{thebibliography}

\newpage

\appendix
\section{Drum Sample Beat Patterns}
To be added
\end{document}



